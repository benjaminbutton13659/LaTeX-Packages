\documentclass[notitlepage, 11pt, a4paper,ngerman]{article}
\usepackage[margin=1in]{geometry}  % changes the margins on all pages (easy lanscape mode)

%German and foreign characters
\usepackage[utf8]{inputenc} % utf8 encoding, umlauts...
\usepackage[T1]{fontenc} % correct searching and copying of umlauts in pdfs
\usepackage[ngerman]{babel} % document germanization of dates and so on

%Graphics
\PassOptionsToPackage{dvipsnames}{xcolor}  % needed to get colors working in certain environments
\usepackage{graphicx} % embed graphics
\graphicspath{ {images/} } % sets one path for all graphics
\usepackage{wrapfig}  % used for wrapping text around a figure (alternative to using minipages)
\PassOptionsToPackage{dvipsnames}{xcolor}  % needed to get colors working in certain environments

%Math
\usepackage{amsmath} % important math environment
\usepackage{amsfonts} % mathmatical fonts
\usepackage{amssymb} % mathmatical symbols
\usepackage{mathtools} % flexible and nice depiction of math
\usepackage{nicefrac}  % adds \nicefrac: a/b fraction display
\usepackage{cancel}  % adds \cancel which adds strikethrough
\usepackage{bm}  % adds \bm: bold font support for math-symbols, replaces outdated \boldsymbol
\usepackage{wasysym} % adds all kinds of symbols

%Links
\usepackage[hidelinks]{hyperref}  % hyperlinks to section or url, hidelinks makes them invisible
\usepackage{cleveref} % smart automatic referenzing
\usepackage{url} % smart handling of url links

%Tikz
\usepackage{tikz} % Tikz ist kein Zeichenprogramm: create vectorgraphics within LaTeX
\usepackage{tikz-cd} % commutative diagrams in tikz
\usetikzlibrary{calc} % allows coordiantes to be calculated
% tikzlibraries add sympols or pathstyles

% Header and Footer:
\usepackage{fancyhdr}
\pagestyle{fancy}
\renewcommand{\sectionmark}[1]{\markright{\thesection.\ #1}}
\renewcommand{\subsectionmark}[1]{#1}
\lhead{\nouppercase{\rightmark}} \chead{} \rhead{\thepage} 
\lfoot{} \cfoot{\thepage} \rfoot{}
\renewcommand{\headrulewidth}{0.2pt}

%Other
\usepackage[shortlabels]{enumitem} % second option in enumerate (a), 1.)...
\usepackage{multirow} % muli
\usepackage{booktabs} % adds \toprule, \midrule, \cmidrule, \bottomrule for nicer tables
\usepackage{tcolorbox} % colored boxes
\usepackage{placeins} % adds \FloatBarrier command to stop floats from floating away
\usepackage{tocloft} % allows manipulation of table of contents

\usepackage{lipsum} % adds exampletext to test Lorem Ipsum...

\hbadness=99999  % removes unnecessary hbadness warnings

% coloring for working at night
%\pagecolor{darkgray}
%\color{white}

\begin{document}

\title{
	{\Huge Pakete Erklärt}\\[1em]
	{\huge Untertitel}\\[1em]
	{\Large Unteruntertitel}}
\author{Author Name}
\date{ \today}
\maketitle

\thispagestyle{empty}
\pagebreak
\pagenumbering{Roman}
\tableofcontents \thispagestyle{fancy}
\listoffigures \thispagestyle{fancy}
\listoftables \thispagestyle{fancy}
\pagebreak

\pagenumbering{arabic}

\newpage

\section{Generelles}

\verb|\documentclass[titlepage, 11pt, a4paper,ngerman]{article}|\\
Setzt die Schriftgröße, das Papierformat und die Dokumentklasse!\\[10pt]
\verb|\usepackage[utf8]{inputenc} % utf8 encoding, umlauts...|\\
\verb|\usepackage[T1]{fontenc}| \texttt{\% correct searching and copying of \ um\-lauts in pdfs}\\
\verb|\usepackage[ngerman]{babel}| \texttt{\% document germanization of dates and so on}\\[10pt]
Umlaut Support: ä (versuche das Dokument danach zu durchsuchen oder die Stelle zu kopieren.)\\
Probiere mal bei babel statt ngerman (new german) greek anzugeben.\\[10pt]
Mathe: $ x^2 - \frac{5}{3}x = 7 $ \label{test}
\begin{equation}
\textrm{Gleichungen:} \quad a^2 + b^2 = c^2
\label{Pythagoras}
\end{equation}
\begin{equation}
c = \sqrt{a^2 + b^2}
\label{2}
\end{equation}

\section{Grafiken}

\subsection{graphicx}

\verb|\usepackage{graphicx} % embed graphics|\\
\verb|\graphicspath{ {images/} } % sets one path for all graphics|\\

\begin{figure}[ht] % here and top aligned
	\centering
	\includegraphics[width=.5\linewidth]{image.png}
	\caption[Bildunterschrift für Verzeichnis]{Bildunterschrift von Grafik 1}
	\label{Bild1}
\end{figure}

\newpage

\subsection{wrapfigure}

\noindent
\verb|\usepackage{wrapfig}| \texttt{\% used for wrapping text around a figure\\ (alternative to using minipages)}\\

\noindent
Beispieltext als Platzhalter:\\
\begin{wrapfigure}{r}{.25\textwidth}
	\centering
	\includegraphics[width=.25\textwidth]{image.png}
	\caption[Bildunterschrift im Verzeichnis 2]{Bildunterschrift in Wrapfigure}	
\end{wrapfigure}
\lipsum[1]
\begin{wrapfigure}{l}{.5\textwidth}
	\centering
	\includegraphics[width=.25\textwidth]{image.png}
	\caption[Bildunterschrift im Verzeichnis 2]{Bildunterschrift in Wrapfigure}	
\end{wrapfigure}
\lipsum[2]

\section{Mathematik}

\verb|\usepackage{amsmath} % important math environment|\\
\verb|\usepackage{amsfonts} % mathmatical fonts|\\
\verb|\usepackage{amssymb} % mathmatical symbols|\\
\verb|\usepackage{mathtools} % flexible and nice depiction of math|\\[10pt]
Viele Grundlegende Sachen die mit der Mathe Umgebung zu tun haben. Symbole, Umgebungen und generell viel Automatisierung und Flexibilität.

\subsection{Nicefrac}

\verb|\usepackage{nicefrac}  % adds \nicefrac: a/b fraction display|\\[10pt]
\begin{equation}
\textrm{Normaler Bruch: } \frac{a}{b} \qquad \textrm{Nicefrac: } \nicefrac{a}{b}
\end{equation}

\subsection{Cancel}

\verb|\usepackage{cancel}  % adds \cancel which adds strikethrough|\\[10pt]
Kürzen in Brüchen (oder anderes Durchstreichen)
\begin{equation}
\frac{\cancel{(a - 5x)} \cdot \sum_{i=0}^{N} c_i}{7 \cancel{\cdot (a - 5x)}}
\end{equation}

\subsection{bm}

\verb|\usepackage{bm}| \texttt{\% adds} \verb|\bm:| \texttt{bold font support for math-symbols, replaces outdated} \verb|\boldsymbol|\\[10pt]
Fette Schrift in Mathe Umgebung $ \bm{[-1;1]} , [-1;1] \ ; \ \bm{\sum_{i=0}^{N}}, \sum_{i=0}^{N}$

\subsection{wasysym}

\verb|\usepackage{wasysym}| \texttt{ \% adds all kinds of symbols}\\[10pt]
\begin{equation}
\twonotes, \leftmoon, \venus, \uranus, \scorpio, \APLlog
\end{equation}

\section{Referenzen}

\subsection{Hyperref}

Fügt die Möglichkeit Hinzu vom Inhaltsverzeichnis zur Seite zu Springen.\\[10pt]
\verb|\usepackage[hidelinks]{hyperref}| \texttt{ \% hyperlinks to section or url, hidelinks makes them invisible}\\[10pt]
\href{https://de.wikipedia.org/wiki/LaTeX}{\LaTeX \ Wikipedia Seite}\\
\href{mailto:test@example.net}{Mail an Test}

\subsection{Normales Referieren}

Wie man in Gleichung \ref{Pythagoras} sieht:

\subsection{Cleverref}

\verb|\usepackage{cleveref}|\\[10pt]
Betrachte die \crefrange{Pythagoras}{2}.\\ Auch \cref{Bild1} ist sehr wichtig.

\subsection{Url}

\verb|\usepackage{url}|\\[10pt]
\url{https://de.wikipedia.org/wiki/LaTeX}


\section{Verschiedenes}

\subsection{Tikz}

\verb|\usepackage{tikz}| \texttt{ \% Tikz ist kein Zeichenprogramm: create vectorgraphics within LaTeX}\\
\verb|\usepackage{tikz-cd}| \texttt{ \% commutative diagrams in tikz}\\
\verb|\usetikzlibrary{calc}| \texttt{ \% allows coordiantes to be calculated}\\[10pt]

Koordinaten-basiertes zeichnen von Vektorgrafiken. Einfach mal Googeln damit kann man wirklich alles Zeichnen: \href{http://www.texample.net/tikz/examples/}{Tikz Example Homepage}
\begin{center}
	\begin{tikzpicture}[scale=0.5]
	\draw[draw=none, fill=NavyBlue] (0,0) -- (-45:3) arc(-45:225:3) -- cycle;
	\draw[draw=none, fill=Goldenrod!90!Brown] (0,0) -- (-65:3) arc(-65:-135:3) -- cycle;
	\draw[draw=none, fill=RawSienna!70!black] (0,0) -- (-45:3) arc(-45:-65:3) -- cycle;
	\coordinate (b) at (5,0);
	\coordinate (a) at ($ (b) + (0,1.5) $);
	\coordinate (c) at ($ (b) - (0,1.5) $);
	\foreach \p\c\t in {a/NavyBlue/Sky, b/Goldenrod!90!Brown/Sunny side of pyramid, c/RawSienna!70!black/Shady side of pyramid}
	\draw[draw=none, fill=\c] (\p) circle(.5cm) node[right=.8cm] {\t};
	\end{tikzpicture}
\end{center}

\subsection{enumitem}

\verb|\usepackage[shortlabels]{enumitem}| \text{ \% second option in enumerate (a), 1.)...}\\[10pt]
Hier im Beispiel mit (a) wird dann (b) und (c) automatisch generiert.

\begin{enumerate}[(a)]
	\item Zählt automatisch weiter
	\item Zahlen wie 1. oder 1) oder (1.)
	\item und auch Buchstaben (a) und a.
\end{enumerate}
Beispiel 2:
\begin{enumerate}[(a)]
	\item Aufzählung 1 Item a
	\item Aufzählung 1 Item b
	\begin{enumerate}[1.)]
		\item Aufzählung 2 Item 1
		\item Aufzählung 2 Item 2
	\end{enumerate}
	\item Aufzählung 1 Item c
\end{enumerate}

\subsection{booktabs}

\verb|\usepackage{booktabs}| \texttt{ \% adds} \verb|\toprule|, \verb|\midrule|, \verb|\cmidrule|, \verb|\bottomrule| \texttt{for nicer tables}\\
\verb|\usepackage{multirow} % muli|\\[10pt]
Schönere Tabellen mit Horizontalen Linien. Hier im Beispiel auch Multirow und Multicolumn.

\begin{table}[h]\centering
	\begin{tabular}{lcccc}
		\toprule 
		\multirow{2}{*}{Sample} & \multicolumn{2}{c}{I} & \multicolumn{2}{c}{II} \\
		\cmidrule(l){2-3} \cmidrule(l){4-5}
		& A & B & C & D \\
		\midrule
		S1 & 5 & 8 & 12 & 2 \\
		S2 & 6 & 9 & 2 & 6 \\
		S3 & 7 & 9 & 5 & 8 \\
		S4 & 8 & 9 & 8 & 2 \\
		\bottomrule
		\end{tabular}
		\caption[Tabellenverzeichnis Eintrag]{Tabellenunterschrift}
\end{table}

\FloatBarrier

\subsection{tcolorbox}

\verb|\usepackage{tcolorbox} % colored boxes|\\[10pt]
Boxen zum Umrahmen von allem. Für Mathe-Umgebungen muss man \verb|\tcboxmath| benutzen.

\begin{tcolorbox}[colframe=red!75!black, title=Überschrift]
	\centering
	Beispieltext
\end{tcolorbox}

\subsection{placeins}

\verb|\usepackage{placeins} % adds \FloatBarrier|\\[10pt]
Um Schwebe-Objekte (Floats) wie Bilder und Tabellen an einer stelle zu fixieren an der sie nicht bleiben wollen kann man im äußersten Notfall den Befehl: \verb|\FloatBarrier| verwenden. Hier können diese nicht vorbeischweben.\par
Zuerst sollte man allerdings die option \verb|[ht]| für here und top aligned ausprobieren.




\end{document}